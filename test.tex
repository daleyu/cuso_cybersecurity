\documentclass[letterpaper,12pt,addpoints]{exam}
\usepackage[utf8]{inputenc}
\usepackage[english]{babel}

\usepackage[top=1in, bottom=1in, left=0.75in, right=0.75in]{geometry}
\usepackage{amsmath,amssymb}

\newcommand{\university}{Columbia University}
\newcommand{\faculty}{Science Olympiad}
\newcommand{\class}{CUSO 2024}
\newcommand{\examnum}{Test \#1}
\newcommand{\content}{Cybersecurity}
\newcommand{\examdate}{2024/01/27}
\newcommand{\timelimit}{50 minutes}

\pagestyle{headandfoot}
\firstpageheader{}{}{}
\firstpagefooter{}{Page \thepage\ of \numpages}{}
\runningheader{\class}{\examnum}{\examdate}
\runningheadrule
\runningfooter{}{Page \thepage\ of \numpages}{}

\begin{document}

\title{\Large \textbf{\university\\ \faculty\\
\bigskip
\class -- \examnum \\ \content}}
\author{Proctors: Geoffrey Wu, Connor Li}
\date{\examdate}
\maketitle
\begin{flushleft}
\makebox[12cm]{\textbf{Name(s)}:\ \hrulefill}
\medskip
\makebox[12cm]{\textbf{School Name/School Code}:\ \hrulefill}
\end{flushleft}
\noindent \rule{\textwidth}{1pt}

\noindent This exam contains \numpages\ pages (including this cover page) and \numquestions\ questions. Total of points is \numpoints.\\
Good luck and Happy reading work!

\begin{center}
\textbf{Distribution of Marks}\\

\gradetable[v][questions]
\end{center}

\clearpage
\begin{questions}
\question
A contractor is required by a county planning department to submit one, two, three, four, or five forms (depending on the nature of the project) in applying for a building permit. Let $Y=$ the number of forms required of the next applicant. The probability that $y$ forms are required is known to be proportional to $y$---that is, $p(y)=ky\ \text{for}\ y=1,\dots,5.$

\begin{enumerate}
    \item Answer the following questions true or false:
    \begin{enumerate}
        \item An operating system can be viewed as a "resource allocator" to control various I/O devices and user programs.
        \item The following instructions must be protected to ensure that a computer system operates correctly: change to monitor mode, read from monitor memory, write into monitor memory, and turn off timer interrupts.
        \item I/O instructions and turning interrupts on are not generally considered to be privileged instructions.
        \item Deadlock can be prevented in the dining philosophers problem by simply reducing the number of philosophers that are allowed to eat at the same time by one.
        \item On a uniprocessor system, the critical section problem can be solved simply by disabling interrupts while a shared variable is being modified.
        \item A thread is generally more lightweight than a process because threads have their own virtual address spaces while processes may have shared address spaces.
        \item A deadlock cannot arise for a set of processes unless there is a circular wait condition.
    \end{enumerate}
    \item You are to implement the OS2000 timer interrupt handler \texttt{tint()} for the timer interrupt mechanism used on the x99 hardware architecture. \ldots [rest of the question as provided]
    \begin{enumerate}
        \item Consider the following \texttt{tint()} implementation. \ldots [rest of the question as provided]
        \item Explain in English a more robust method than the one used \ldots [rest of the question as provided]
        \item You are now told that due to a new change in x99, there are high priority interrupts \ldots [rest of the question as provided]
    \end{enumerate}
    \item Give short answers to each of the following questions.
    \begin{enumerate}
        \item What is the difference between \texttt{fork} and \texttt{clone} in Linux?
        \item What is the difference between maskable and non-maskable interrupts?
        \item What is the difference between a thread and a process?
        \item What is the difference between kernel mode and user mode?
    \end{enumerate}
    \item Write a new Linux 5.10.138 system call \texttt{pinfo()} that takes a PID and a pointer to \texttt{struct proc\_struct} as its arguments, and populates the structure with the process state information for that process. The system call should be numbered 441. \ldots [rest of the question as provided]
    \begin{enumerate}
        \item Write the C code for your system call function \ldots [rest of the question as provided]
        \item Write a simple C program that calls your system call and prints the result. \ldots [rest of the question as provided]
    \end{enumerate}
\end{enumerate}

\clearpage
\question
The \textit{pmf} of the amount of memory \textit{X} (GB) in a purchased flash drive was given as
\begin{center}
\begin{tabular}{c|c|c|c|c|c} 
 $x$ & 1 & 2 & 4 & 8 & 16 \\
 \hline
 $p(x)$ & 0.05 & 0.10 & 0.35 & 0.40 & 0.10\\  
\end{tabular}
\end{center}
Compute the following:
\begin{parts}
\part[2] Expected value $E(X)$
\part[2] Variance $V(X)$ directly from the definition
\part[2] The standard deviation $\sigma(X)$
\part[2] $V(X)$ using the shortcut formula ($V(X)=E(X^2)-E^2(X)$)
\end{parts}

\clearpage
\question Each of 12 refrigerators of a certain type has been returned to a distributor because of the presence of a high-pitched oscillating noise when the refrigerator is running. Suppose that 5 of these 12 have defective compressors and the other 7 have less serious problems. If they are examined in random order, let X = the number among the first 6 examined that have a defective compressor. Compute the following:
\begin{parts}
\part[3] $P(X=1)$
\part[3] $P(X \geq 4)$
\end{parts}

\clearpage
\question A reservation service employs five information operators who receive requests for information independently of one another, each according to a Poisson process with rate $\mu=2$ per minute.
\begin{parts}
\part[4] What is the probability that during a given 1-min period, the first operator receives no requests?
\part[4] What is the probability that during a given 1-min period, exactly four of the five operators receive no requests?(\textit{Hint}: treat either as a binomial process of 5 trials with 4 successes or consider 5 combinations of Poisson processes, e.g. only 1st operation receives a request  or only 2nd operation receives a request and so on)
\end{parts}

\end{questions}
\clearpage

\centering \textbf{\large Probability mass/distribution functions}

\flushleft \textbf{Binomial Distribution}
$$f(x;n,p)=b(x;np)=\binom{n}{x}p^x(1-p)^{n-x}$$
$$\mu=E(x)=np$$
$$\sigma^2_x=np(1-p)$$

\flushleft \textbf{Hypergeometric Distribution}
$$P(X=x)=h(x;n,M,N)=\frac{\binom{M}{x}\binom{N-M}{n-x}}{\binom{N}{n}}$$
$$\mu=E(X)=\frac{nM}{N}$$
$$\sigma^2_x=n\frac{M}{N}\frac{N-M}{N}\frac{N-n}{N-1}$$

\flushleft \textbf{Poisson Distribution}
$$P(x;\mu)=e^{-\mu}\frac{\mu^x}{x!}$$
$$E(X)=Var(X)=\mu$$

\clearpage
This page is intentionally left blank to accommodate work that wouldn't fit elsewhere and/or scratch work.
\end{document}
