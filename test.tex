\documentclass[letterpaper,12pt,addpoints]{exam}
\usepackage[utf8]{inputenc}
\usepackage[english]{babel}
\usepackage{hyperref}

\usepackage[top=1in, bottom=1in, left=0.75in, right=0.75in]{geometry}
\usepackage{amsmath,amssymb}

\newcommand{\university}{Columbia University}
\newcommand{\faculty}{Science Olympiad}
\newcommand{\class}{CUSO 2024}
\newcommand{\examnum}{Test \#1}
\newcommand{\content}{Cybersecurity}
\newcommand{\examdate}{2024/01/27}
\newcommand{\timelimit}{50 minutes}

\pagestyle{headandfoot}
\firstpageheader{}{}{}
\firstpagefooter{}{Page \thepage\ of \numpages}{}
\runningheader{\class}{\examnum}{\examdate}
\runningheadrule
\runningfooter{}{Page \thepage\ of \numpages}{}

\begin{document}

\title{\Large \textbf{\university\\ \faculty\\
\bigskip
\class -- \examnum \\ \content}}
\author{Proctors: Geoffrey Wu, Connor Li}
\date{\examdate}
\maketitle
\begin{flushleft}
\makebox[12cm]{\textbf{Name(s)}:\ \hrulefill}
\medskip
\makebox[12cm]{\textbf{School Name/School Code}:\ \hrulefill}
\end{flushleft}
\noindent \rule{\textwidth}{1pt}

\noindent This exam contains 4 pages (including this cover page) and 4 sections. Total of points is 100.\\
Good luck and Happy reading work!

\begin{center}
\textbf{Distribution of Marks}\\
\end{center}
\begin{center}
\begin{tabular} {| c | c | c |}
    \hline
    Section & Points & Score \\
    \hline
    1 & 40 & \\
    \hline
    2 & 20 & \\
    \hline
    3 & 20 & \\
    \hline
    4 & 20 & \\
    \hline
\end{tabular}
\end{center}
\clearpage

\section*{Section 1 (40 points)}
\subsubsection*{ECC (8 points)}
$[1]$ Write the general form of an elliptic curve. (2 points).\\
\\
$[2]$ Consider the following elliptic curve $y^2 \equiv x^3 + 2x + 3\;(\text{mod } 5)$. 
\begin{itemize}
    \item Find the number of points on the curve. List those points. (3 points).
    \item Find the order of the curve. (1 point).
    \item Find a generator of the curve. (1 point).
\end{itemize}
$[3]$ Using the EC $y^2 \equiv x^3 - x + 4 \;(\text{mod } 5)$, find the following EC addition $(2,0) + (4,3)$. (1 point).
\subsubsection*{Diffie-Hellman Key Exchange (7 points)}
$[1]$ True or False. The Diffie Hellman key exchange is a building block for the SSH and SSL/TLS protocols. (1 point).\\
\\
$[2]$ Determine and fill in the missing step in the below key exchange (1 point per question, 5 points total).
\begin{enumerate}
    \item Alice and Bob agree on two large prime numbers, $p$ and $g$, and a public key exchange algorithm.
    \item Alice chooses a secret integer, $a$, and computes $A = g^a \mod p$. She sends $A$ to Bob.
    \item Bob chooses a secret integer, $b$, and computes $B = g^b \mod p$. He sends $B$ to Alice.
    \item Alice computes $s= \underline{\hspace{2cm}}$. Bob computes $s= \underline{\hspace{2cm}}$.
    \item Alice and Bob now both have the shared secret key $s$, which they can use to establish a secure communication channel.
\end{enumerate}
$[3]$ Name one vulnerability of this type of key exchange. (1 point).
\subsubsection*{Hashing (8 points)}
$[1]$ What is hash collision? (2 points).\\
\\
$[2]$ Explain the difference between Hash Function and Encryption. (2 points).\\
\\
$[3]$ True or False. A hash function is a one-way function. (1 point).\\
\\
$[4]$ Write a piece of code to create a simple Hash Function converting each character in the input key into its corresponding ASCII value. (3 points).
\subsubsection*{XOR (7 points)}
$[1]$ Fill in the following XOR table (4 points).
$$
\begin{tabular}{|c | c | c|}
    \hline
    x & y & x $\oplus$ y \\
    \hline 
    0 & 0 & \\
    \hline
    0 & 1 & \\
    \hline
    1 & 0 & \\
    \hline
    1 & 1 & \\
    \hline
\end{tabular}
$$
$[2]$ Swap two values $x$ and $y$ in-place, i.e. without using any helper variables and using the XOR function. (3 point).
\subsubsection*{Classical Ciphers (3 points)}
$[1]$ True or False. Monoalphabetic ciphers are stronger than polyalphabetic ciphers because frequency analysis is tougher on the former. (1 point).\\
\\
$[2]$ Is the Caesar Cipher is an example of a monoalphabetic or a polyalphabetic cipher? (1 point).\\
\\
$[3]$ Use Caesar Cipher to encrypt the following plaintext with key $k=10$. (1 point).
$$
\text{Columbia Science Olympiad} \implies ?
$$
\subsubsection*{Modern Ciphers (7 points)}
$[1]$ True or False. Rivest-Shamir-Adleman, or RSA, is an algorithm used for symmetric key cryptography. (1 point).\\
\\
$[2]$ In password protection, what is the name of a random string of data used to modify a password hash called? (1 point).\\
\\
$[3]$ One of the triumphs of modern cryptanalysis was the breaking of what cipher used by German U-Boats during World War II? (1 point).\\
\\
$[4]$ CHELSEA and ARSENAL are both $700$ kms away from you. SPURS $1800$ kms away from you. WOLVES $200$ kms away from you. How far are from MANCITY? (2 points).\\
\\
$[5]$ What is the units digit of $6789^5 \mod 5$? (2 points).

\pagebreak
\section*{Section 2 (20 points)}
\subsubsection*{Web Architecture (12 points)}
$[1]$ What is an API and what does it do? (1 point).\\
\\
$[2]$ Name the $4$ most common HTTP methods (verbs). (4 points).\\
\\
$[3]$ What is the difference between HTTP and HTTPS? (2 points).\\
\\
$[4]$ What are 200 OK response codes in HTTP? (1 point).\\
\\
$[5]$ Name the 3 parts of a HTTP response. (3 points).\\
\\
$[6]$ What is the role of Public Key Infrastructure (PKI) in HTTPS? (1 point).


\subsubsection*{Cybersecurity Principles (8 points)} 
$[1]$ A password, a physical token and iris scan are common methods of what? (1 point).\\
\\
$[2]$ Alice, using her private key encryption, created an email to Bob which provides a digital signature authenticating that it was Alice sending an email to Bob. Alice realizes she cannot take back her email and claim it was never sent by her. What is this an example of? (1 point).\\
\\
$[3]$ Describe what a DDoS attack is. (1 point). \\
\\
$[4]$ What is Kerckhoff’s principle? (1 point).\\
\\
$[5]$ Name the a main consequence of a cyber attack for an organization. (1 point).\\
\\
$[6]$ Explain CIA triad. (1 point).\\
\\
$[7]$ Name a common cyberattack. (1 point).\\
\\
$[8]$ Name two layers of the OSI model. (1 point).

\section*{Section 3 (20 points)}
Data Structures Problem. 
\section*{Section 4 (20 points)}
Recursion-Focused Problem. 

\end{document}

